\chapter{Simetrías y reglas de conservación}

\section{}

\section{}

\section{Grupo no-conmutativo: las rotaciones y el momentum angular}

Aplicar propiedades generales a problemas específicos, pero el grupo de rotaciones no es conmutativo.

Suponer un hamiltoniano $\hat{H}$ invariante ante rotaciones:

\begin{align*}
    \left[\hat{R}_{\vec{\mu}}(\alpha),\hat{H}\right]&=0\\ 
    \left[\hat{R}_{\vec{\mu}}(\alpha),\hat{U}(t)\right]&=0
\end{align*}

Tomamos una rotación de un ángulo $\alpha$ al rededor de $z$:

\begin{align*}
    \phi'(r,\theta,\phi)
        &=\hat{R}_z(\alpha)\phi(r,\theta,\phi)\\
        &=\phi(r,\theta,\phi+\alpha)
\end{align*}

Por definición de generador, el generador $\hat{L}_z$ es tal que $\hat{R}_z(\alpha)=\exp\left(-\frac{i}{\hbar}\hat{L}_z\alpha\right)$.

\subsection{Escritura vectorial}

\begin{align*}
    \hat{\vec{L}}=\hat{\vec{r}}\times\hat{\vec{p}}=
    \begin{cases}
        \hat{L}_x=\hat{y}\hat{P}_z-\hat{z}\hat{P}_y=-i\hbar(y\dd z-z\dd y)\\
        \hat{L}_y=\hat{z}\hat{P}_x-\hat{x}\hat{P}_z=-i\hbar(z\dd x-x\dd z)\\
        \hat{L}_z=\hat{x}\hat{P}_y-\hat{x}\hat{P}_y=-i\hbar(x\dd y-y\dd x)
    \end{cases}
\end{align*}

\subsection{Relaciones de conmutación}

Las relaciones de conmutación de $\hat{\vec{L}}$ están dadas por:

\begin{equation}
    \left[\hat{L}_i,\hat{L}_j\right]=i\hbar\varepsilon_{ijk}\hat{L}_k
\end{equation}

\subsection{Espacio de representaciones}

Un espacio invariante ante un grupo de transformaciones, se le llama espacio de representaciones del grupo.
Puede ser reducible si existe una suma directa de subespacios que sean invariantes ante este mismo grupo.

\subsubsection{Espacio de representaciones irreducibles de un grupo conmutativo}

\textbf{Propiedad:} Todas las representaciones irreducibles son de dimensión 1.

\subsubsection{Espacio de representaciones irreducibles de los grupos de rotación $SU(2)$ y  $SO(3)$}

Nos vamos a interesar en los generadores de rotaciones (operadores hermíticos y unitarios).

\textbf{Propiedad:}
Sean $\hat{J_x},\hat{J_y},\hat{J_z}$  operadores autoadjuntos en un espacio de Hilbert $\mathbb{H}$ que verifican el álgebra de Lie del grupo de rotaciones:

\begin{align}
    \left[J_i,J_j\right]=i\hbar\varepsilon_{ijk}\hat{J}_k
\end{align}

entonces ${\hat{\vec{J}}}^2$ conmuta con los elementos del álgebra de Lie.

\begin{align*}
    \left[{\hat{\vec{J}}}^2,\hat{J_i}\right]=0 \text{ para } i=x,y,z
\end{align*}

El espectro común de $J_z$ y $\vec{J}^2$ es de la forma:

\begin{align}
    \vec{J}^2\ket{j,m}&=\hbar^2j(j+1)\ket*{j,m}\\
    J_z\ket*{j,m}&=
\end{align}

Definimos los operadores escalón:

\begin{align*}
    J_+&=J_x+iJ_y\\
    J_-&=J_x-iJ_y
\end{align*}

Y notamos que $J_+^\dagger=J_-$

Las relaciones de conmutación:

\begin{align*}
    \left[J_+,J_-\right]&=2\hbar J_z\\
    \left[J_z,J_\pm\right]&=\pm \hbar J_{pm}\\
    \left[\vec{J}^2,J_{\pm}\right]&=0
\end{align*}

\medskip


Con $j$ fijo, los vectores $\ket*{j,m}$, con $m=-j,\cdots,j$ forman una base ortonormal $(\bra*{j,m}\ket*{j,n}=\delta_{mn})$ de un espacio vectorial $D_j$ de dimensión $j+1$ y el espacio $D_j$ es un espacio de representaciones irreducibles del grupo de rotación.

Todo espacio de representaciones del grupo de rotaciones puede descomponerse como la suma de espacios irreducibles $$H=D_{j1} \oplus D_{j2} \oplus\cdots$$ (con índices que pueden repetirse).

\bigskip

\hspace{1cm}\textbf{Nota:} Los 3 generadores $J_x,J_y,J_z$ forman una base del álgebra de Lie llamada álgebra de Lie $SO(3)$ o $SU(2)$. Los generadores $J_x,J_y,J_z$ forman otra base de la misma álgebra más interesante. A esta base se le llama descomposición de Cartan de $SU(2)$, $SO(3)$ complejificada.

Tenemos que:

\begin{align*}
    J_z\left(J_\pm \ket*{a,m}\right)
        &=\left(\left[J_z,J_\pm\right]+J_\pm J_z\right)\ket*{a,m}\\
        &=\pm \hbar J_\pm \ket*{a,m}+\hbar mJ_\pm\ket*{a,m}\\
        &=\hbar(\pm1+m)J_\pm\ket*{a,m}
\end{align*}

por otro lado,

\begin{align*}
    \vec{J}^2\left(J_\pm\ket*{a,m}\right)=J_\pm \vec{J}^2 \ket*{a,m}=\hbar^2aJ_\pm\ket*{a,m}
\end{align*}

\textbf{Nota:}

\begin{align*}
    \vec{J}^2=\frac{1}{2}\left(J_+J_-+J_-J_+\right)J_z^2
\end{align*}

Entonces notemos que:

\begin{align*}
    \hbar^2(a-m^2)
        &=\bra{a,m}\vec{J}^2-J_z^2\ket{a,m}\\
        &=-\frac{1}{2}\bra{a,m}\left(J_+J_-+J_-J_+\right)\ket{a,m}\\
        &\geq 0\\
    a&\geq m^2
\end{align*}

Existe un valor máximo de $m:m_{max}$, y por tanto un mínimo. 

Esto se traduce en:

\begin{align*}
    J_+\ket{a,m_{max}}&=0\\
    J_-\ket{a,m_{min}}&=0
\end{align*}

Por otro lado,

\begin{align*}
    \vec{J^2}
        &=J_z^2+\frac{1}{2}\left(J_+J_-+J_-J_+\right)\\
        &=J_z^2+J_+J_--\hbar J_z
\end{align*}

Aplicamos la relación a $\ket{a,m_{max}}$ y a $\ket{a,m_{min}}$:

\begin{align*}
    \vec{J}^2\ket{a,m_{max}}
        &=\hbar^2a^2\ket{a,m_{max}}\\
        &=\\
        &=\hbar^2 m_{max} (m_{max}+1)\ket{a,m_{max}}
\end{align*}

$$a^2=m_{max}(m_{max}+1)$$

de igual manera $a^2=m_{min}(m_min-1)$.

\medskip

Dado que $J_z\left(J_{\pm}\ket{a,m}\right)=\hbar(\pm1+m)(J_\pm\ket{a,m})$ podemos definir:

\begin{align}
    \ket{a,m=m\pm1}=\frac{1}{c_\pm}J_\pm\ket{a,m}
\end{align}

podemos intuir que $m_{max}=m_{min}+n$ con $n\in \mathbb{N}$ entonces:

\begin{align}
    a^2=m^2_{max}+m_{max}=m_{min}^2+m_{min}
\end{align}

De esto: $m_{min}=-\frac{n}{2}$ y por tanto $m_{max}=m_{min}+n=\frac{n}{2}=j$.

Por tanto $m=-j,-j+1,\cdots,j-1,j$ y tenemos $(2j+1)$ valores posibles.

Ahora la normalización $c_\pm$, si $\ket{j,m}$ está normalizado,

\begin{align*}
    \abs{c_{\pm}}^2\norm*{\ket{j,m+1}}^2
        &=\bra{j,m}J_-J_+\ket{j,m}\\
        &=\hbar^2 \left(j(j+1)-m(m+1)\right)
\end{align*}

de igual manera para $c_-$.

Por tanto, cada espacio $D_j$ es de dimensión $(2j+1)$ y los operadores $J_\pm$, $J_z$ actúan dentro de $D_j$ al igual que $J_x,J_y,J_z$ que se obtienen por medio de combinaciones lineales. 

$D_j$ es invariante ante rotaciones: es un espacio de representación del grupo de rotación.

Cada $D_j$ no puede descomponerse en la suma de 2 espacios. 

\hfill$\qed$