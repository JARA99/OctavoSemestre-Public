\chapter{Simetrías y reglas de conservación}

\section{}

\section{}

\section{Grupo no-conmutativo: las rotaciones y el momentum angular}

Aplicar propiedades generales a problemas específicos, pero el grupo de rotaciones no es conmutativo.

Suponer un hamiltoniano $\hat{H}$ invariante ante rotaciones:

\begin{align*}
    \left[\hat{R}_{\vec{\mu}}(\alpha),\hat{H}\right]&=0\\ 
    \left[\hat{R}_{\vec{\mu}}(\alpha),\hat{U}(t)\right]&=0
\end{align*}

Tomamos una rotación de un ángulo $\alpha$ al rededor de $z$:

\begin{align*}
    \phi'(r,\theta,\phi)
        &=\hat{R}_z(\alpha)\phi(r,\theta,\phi)\\
        &=\phi(r,\theta,\phi+\alpha)
\end{align*}

Por definición de generador, el generador $\hat{L}_z$ es tal que $\hat{R}_z(\alpha)=\exp\left(-\frac{i}{\hbar}\hat{L}_z\alpha\right)$.

\subsection{Escritura vectorial}

\begin{align*}
    \hat{\vec{L}}=\hat{\vec{r}}\times\hat{\vec{p}}=
    \begin{cases}
        \hat{L}_x=\hat{y}\hat{P}_z-\hat{z}\hat{P}_y=-i\hbar(y\dd z-z\dd y)\\
        \hat{L}_y=\hat{z}\hat{P}_x-\hat{x}\hat{P}_z=-i\hbar(z\dd x-x\dd z)\\
        \hat{L}_z=\hat{x}\hat{P}_y-\hat{x}\hat{P}_y=-i\hbar(x\dd y-y\dd x)
    \end{cases}
\end{align*}

\subsection{Relaciones de conmutación}

Las relaciones de conmutación de $\hat{\vec{L}}$ están dadas por:

\begin{equation}
    \left[\hat{L}_i,\hat{L}_j\right]=i\hbar\varepsilon_{ijk}\hat{L}_k
\end{equation}

\subsection{Espacio de representaciones}

Un espacio invariante ante un grupo de transformaciones, se le llama espacio de representaciones del grupo.
Puede ser reducible si existe una suma directa de subespacios que sean invariantes ante este mismo grupo.

\subsubsection{Espacio de representaciones irreducibles de un grupo conmutativo}

\textbf{Propiedad:} Todas las representaciones irreducibles son de dimensión 1.

\subsubsection{Espacio de representaciones irreducibles de los grupos de rotación $SU(2)$ y  $SO(3)$}

Nos vamos a interesar en los generadores de rotaciones (operadores hermíticos y unitarios).

\textbf{Propiedad:}
Sean $\hat{J_x},\hat{J_y},\hat{J_z}$  operadores autoadjuntos en un espacio de Hilbert $\mathbb{H}$ que verifican el álgebra de Lie del grupo de rotaciones:

\begin{align}
    \left[J_i,J_j\right]=i\hbar\varepsilon_{ijk}\hat{J}_k
\end{align}

entonces ${\hat{\vec{J}}}^2$ conmuta con los elementos del álgebra de Lie.

\begin{align*}
    \left[{\hat{\vec{J}}}^2,\hat{J_i}\right]=0 \text{ para } i=x,y,z
\end{align*}

El espectro común de $J_z$ y $\vec{J}^2$ es de la forma:

\begin{align}
    \vec{J}^2\ket{j,m}&=\hbar^2j(j+1)\ket*{j,m}\\
    J_z\ket*{j,m}&=
\end{align}

Definimos los operadores escalón:

\begin{align*}
    J_+&=J_x+iJ_y\\
    J_-&=J_x-iJ_y
\end{align*}

Y notamos que $J_+^\dagger=J_-$

Las relaciones de conmutación:

\begin{align*}
    \left[J_+,J_-\right]&=2\hbar J_z\\
    \left[J_z,J_\pm\right]&=\pm \hbar J_{pm}\\
    \left[\vec{J}^2,J_{\pm}\right]&=0
\end{align*}

\medskip


Con $j$ fijo, los vectores $\ket*{j,m}$, con $m=-j,\cdots,j$ forman una base ortonormal $(\bra*{j,m}\ket*{j,n}=\delta_{mn})$ de un espacio vectorial $D_j$ de dimensión $j+1$ y el espacio $D_j$ es un espacio de representaciones irreducibles del grupo de rotación.

Todo espacio de representaciones del grupo de rotaciones puede descomponerse como la suma de espacios irreducibles $$H=D_{j1} \oplus D_{j2} \oplus\cdots$$ (con índices que pueden repetirse).

\bigskip

\hspace{1cm}\textbf{Nota:} Los 3 generadores $J_x,J_y,J_z$ forman una base del álgebra de Lie llamada álgebra de Lie $SO(3)$ o $SU(2)$. Los generadores $J_x,J_y,J_z$ forman otra base de la misma álgebra más interesante. A esta base se le llama descomposición de Cartan de $SU(2)$, $SO(3)$ complejificada.

Tenemos que:

\begin{align*}
    J_z\left(J_\pm \ket*{a,m}\right)
        &=\left(\left[J_z,J_\pm\right]+J_\pm J_z\right)\ket*{a,m}\\
        &=\pm \hbar J_\pm \ket*{a,m}+\hbar mJ_\pm\ket*{a,m}\\
        &=\hbar(\pm1+m)J_\pm\ket*{a,m}
\end{align*}

por otro lado,

\begin{align*}
    \vec{J}^2\left(J_\pm\ket*{a,m}\right)=J_\pm \vec{J}^2 \ket*{a,m}=\hbar^2aJ_\pm\ket*{a,m}
\end{align*}

\textbf{Nota:}

\begin{align*}
    \vec{J}^2=\frac{1}{2}\left(J_+J_-+J_-J_+\right)J_z^2
\end{align*}

Entonces notemos que:

\begin{align*}
    \hbar^2(a-m^2)
        &=\bra{a,m}\vec{J}^2-J_z^2\ket{a,m}\\
        &=-\frac{1}{2}\bra{a,m}\left(J_+J_-+J_-J_+\right)\ket{a,m}\\
        &\geq 0\\
    a&\geq m^2
\end{align*}

Existe un valor máximo de $m:m_{max}$, y por tanto un mínimo. 

Esto se traduce en:

\begin{align*}
    J_+\ket{a,m_{max}}&=0\\
    J_-\ket{a,m_{min}}&=0
\end{align*}

Por otro lado,

\begin{align*}
    \vec{J^2}
        &=J_z^2+\frac{1}{2}\left(J_+J_-+J_-J_+\right)\\
        &=J_z^2+J_+J_--\hbar J_z
\end{align*}

Aplicamos la relación a $\ket{a,m_{max}}$ y a $\ket{a,m_{min}}$:

\begin{align*}
    \vec{J}^2\ket{a,m_{max}}
        &=\hbar^2a^2\ket{a,m_{max}}\\
        &=\\
        &=\hbar^2 m_{max} (m_{max}+1)\ket{a,m_{max}}
\end{align*}

$$a^2=m_{max}(m_{max}+1)$$

de igual manera $a^2=m_{min}(m_min-1)$.

\medskip

Dado que $J_z\left(J_{\pm}\ket{a,m}\right)=\hbar(\pm1+m)(J_\pm\ket{a,m})$ podemos definir:

\begin{align}
    \ket{a,m=m\pm1}=\frac{1}{c_\pm}J_\pm\ket{a,m}
\end{align}

podemos intuir que $m_{max}=m_{min}+n$ con $n\in \mathbb{N}$ entonces:

\begin{align}
    a^2=m^2_{max}+m_{max}=m_{min}^2+m_{min}
\end{align}

De esto: $m_{min}=-\frac{n}{2}$ y por tanto $m_{max}=m_{min}+n=\frac{n}{2}=j$.

Por tanto $m=-j,-j+1,\cdots,j-1,j$ y tenemos $(2j+1)$ valores posibles.

Ahora la normalización $c_\pm$, si $\ket{j,m}$ está normalizado,

\begin{align*}
    \abs{c_{\pm}}^2\norm*{\ket{j,m+1}}^2
        &=\bra{j,m}J_-J_+\ket{j,m}\\
        &=\hbar^2 \left(j(j+1)-m(m+1)\right)
\end{align*}

de igual manera para $c_-$.

Por tanto, cada espacio $D_j$ es de dimensión $(2j+1)$ y los operadores $J_\pm$, $J_z$ actúan dentro de $D_j$ al igual que $J_x,J_y,J_z$ que se obtienen por medio de combinaciones lineales. 

$D_j$ es invariante ante rotaciones: es un espacio de representación del grupo de rotación.

Cada $D_j$ no puede descomponerse en la suma de 2 espacios. 

\hfill$\qed$

\subsection{En resumen: Representaciones irreducibles de los grupos de rotación $SO(3)$ y $SU(2)$}

Recordemos que:

\begin{align*}
    \vec{J}^2\ket{j,m}=\hbar^2j(j+1)\ket{j,m}\\
    J_z\ket{j,m}=\hbar m\ket{j,m}
\end{align*}

Vamos a tomar ahora al estado $\ket{j,m}$ y lo vamos a rotar por $2\pi$ al rededor de $z$.

\begin{align*}
    R_z(2\pi)
        &=\pm\ket{j,m}
\end{align*}

$+$ si $m$ es entero, y $-$ si $m$ es medio entero.

Entonces:

Si $D_j$ es una representación de $SO(3)$, entonces $R(2\pi)=\mathbb{I}$ $\Rightarrow j$ (y por tanto $m$ también) debe ser un entero.

$$
\rightarrow l=j=0,1,2,\cdots
$$

así, el espacio de representaciones irreducibles de $SO(3)$ son los espacios $D_l$ caracterizados por el entero $l$. Notemos que $D_l$ es de dimensión impar $(2l+1)$.

Si $D_j$ es una representación de $SU(2)$ (rotación de espín $1/2$) $\Rightarrow R(4\pi)=\mathbb{I}$, entonces los valores de $j$ permitidos son:

$$
j=0,1/2,1,\cdots
$$

Los espacios de representación irreducibles de $SU(2)$ son los espacios $D_j$ caracterizados por el entero o medio entero $j$.

\textbf{Nota:}
\begin{itemize}
    \item Del hecho que $\left[\vec{J}^2,\vec{J}\right]=0\Rightarrow\left[J^2,R\right]=0$ para cualquier operador de rotación.
    $$
    R_\mu (\theta)=\exp()
    $$
    Decimos que $\vec{J}^2$ es un operador de Casimir del grupo de rotación.
    Uno de los espacios de representación irreducibles $D_j$ es el espacio propio de $\vec{J}^2$ caracterizado por el autovalor $\hbar^2 $

    \item Los vectores $\ket{j,m}$ forman una base de $D_j$. Estos vectores son autovectores de $J_z$ y la dirección de esta base depende de la elección del eje $z$
\end{itemize}

\textbf{Ejemplos}

\begin{enumerate}
    \item Para describir el espín $1/2$, el espacio $\mathbb{H}_{espin}=\mathbb{C}^2$ de dimensión 2 y los operadores $S_x,S_y,S_z$ y se identifican con el espacio $D_{1/2}$ para $j=1/2$.
    \item El espacio $\mathbb{R}^3$ es un espacio de representaciones irreducible de $SO(3)$. $\mathbb{R}^3$ se identifica al espacio $D_{l=1}$ con base $\ket{l=1,m=-1,0,+1}$ que identificamos con $\ket{x},\ket{y},\ket{z}$ de la siguiente manera:
    \begin{enumerate}
        \item vamos a escribir
    \end{enumerate}
\end{enumerate}


\subsubsection{Aplicación: espectro de un rotor rígido:}

Tomar el ejemplo de una molécula diatómica $(HCl,NaCl,H_2,O_2,\cdots)$, vamos a ignorar la vibración de la molécula y la traslación $\Rightarrow$ solo consideramos rotaciones en ella misma:

\includegraphics[width=0.4\textwidth]{Graficas/G1-Aug2.png}

\textbf{El espacio de Hilbert de los estados cuánticos:}

Movimiento reducido a una esfera $S^2$

$$
\mathbb{H}=L^2(S^2)
$$

los vectores son funciones de onda $\Psi(\theta,\phi)$ en un punto $(\theta,\phi)$ sobre la esfera.
La esfera es el espacio de configuraciones del sistema.

Denotamos por $\ket{\theta,\phi}$ al estado de la posición (delta de Dirac en el punto).

$$
\psi(\theta,\phi)=\bra{\theta,\phi}\ket{\psi}
$$

y la resolución de la identidad:

Donde $\dd\Omega$ es el ángulo sólido de la esfera.

\textbf{Operador Hamiltoniano:}

En el caso clásico:

\begin{align*}
    H=\frac{1}{2}\mu v^2=\frac{1}{2} \mu r_o^2\omega^2=\frac{L^2}{2I}
\end{align*}

con $I$ el momento de inercia $I=\mu r_o^2$ y $L$ el momentum angular.

Cuantizamos 
$$
\hat{H}=\frac{\hat{I}^2}{2I}
$$

con $\vec{L}^2=L_x^2+L_y^2+L_z^2$

\textbf{Espectro de $\hat{H}$}

Teorema: En el espacio de hilbert $\mathbb{H}=L^2(S^2)$, cada espacio de representaciones irreducibles $D_l$, $l=0,1,2,\cdots$ aparece solo una vez:

\begin{align*}
    \mathbb{H}=L^2(s^2)=\bigoplus_{l=0}^\infty D_l
\end{align*}


\subsection{Importancia de las representaciones irreducibles:}

\subsubsection{Propiedades fundamentales: el lema de Schur y el teorema de Wigner}

El lema de Schur tiene varias formulaciones, pero la más útil en mecánica cuántica es la siguiente:

\textbf{Lema de Schur:} Sea $\hat{A}:\mathbb{H}\rightarrow\mathbb{H}$ un operador que actúa sobre el espacio de Hilbert $\mathbb{H}$. Vamos a suponer que este es la suma directa de espacios irreducibles de un grupo $G$, que pos simplicidad diremos que son $2$.

\begin{equation*}
    \mathbb{H}=D_j\oplus D_k\tag{1}
\end{equation*}

y estas representaciones no son equivalentes:
$$
D_j\neq D_k
$$
Bien $j\neq k$ y además para todo $\hat{G}\in G$
$$
\left[A,G\right]=0
$$
$g$ es un grupo de simetrías de $\hat{A}$. Entonces $\hat{A}$ se escribe respecto a la descomposición $(1)$ como:

$$
\hat{A}=\begin{pmatrix}
    a_j\mathbb{I} & 0\\
    0 & a_k\mathbb{I}
\end{pmatrix}
$$

(el resultado es generalizable a la suma de más representaciones irreducibles)

\textbf{Prueba:} Para esto necesitaremos 2 lemas intermedios:

\textbf{Lema de Schur A:} Si $\hat{A}:D_j\rightarrow D_k$ es un operador lineal entre 2 espacios de representaciones irreducibles de $G$ y $\left[A,G\right]=0$ $\forall \hat{G}\in G$ entonces $A=0$ o bien $A$ es un isomorfismo $\Rightarrow D_j\approxeq D_k,j=k$

Nota: 
\includegraphics[width=0.4\textwidth]{Graficas/G2-Aug2.png}

\textbf{Prueba:} El Kernel de $A$, $\ker A\subset D_j$ es invariante por $G$. En efecto, si $\psi\in\ker A$, i.e. $A\psi=0$ entonces sean $\psi '=G\psi$ entonces:

\begin{align*}
    A\psi ' =AG\psi=GA\psi=0\Rightarrow\psi ' \in\ker A
\end{align*}

Ahora bien, $D_j$ es irreducible $\Rightarrow \ker A=\{0\}\Rightarrow A$ es inyectiva o $A=0$.

Por otro lado, $Im(A)\subset D_k$ es invariante por $G$. En efecto, si $\psi \in Im(A)$, $\psi=A\phi\Rightarrow\psi'\in Im(A)$ Como $D_k$ es irreducible, $Im(A)=0$ o $Im(A)=D_k$ (es sobreyectiva).

Por tanto, $A=0$ o $A$ es isomorfismo.

\textbf{Lema de Schur B:}

Si $A:D_j\rightarrow D_k$ es un operador lineal en un espacio de representaciones irreducibles de $G$ y $\left[A,G\right]=0 \forall \hat{G}\in G$ entonces $A=a\mathbb{I}$ con $a\in \mathbb{C}$

\textbf{Prueba:}
Los espacios propios de $A$ denotados $\mathbb{H}_a\subset D_j$ son invariantes por la acción de $G$. Pero $D_j$ es irreducible. Entonces solo hay uno.

El resto de la prueba se basa en escribir al operador $A$ como una matriz de $2\times 2$ utilizando proyecciones, donde los elementos de matriz $A_{ab}$ están dados por $P_a A p_b$. Y por tanto:

\begin{align*}
    \begin{pmatrix}
        a_j \mathbb{I} & 0\\
        0 & a_k \mathbb{I}
    \end{pmatrix}
\end{align*}


\textbf{Teorema de Wigner}
Si el hamiltoniano $\hat{H}$ tiene un espectro discreto y admite una simetría por un grupo de invarianza G discreto o continuo, i.e.$$\left[H,G\right]=0\forall G\in G$$

entonces genéricamente (el resultado de toda perturbación respecto a esta simetría es estable), los espacios propios de $H$ son representaciones irreducibles de $G$.

En particular, si el grupo $G$ es conmutativo, sus representaciones irreducibles son de dimensión $1$ y esperaríaos que no haya degeneración en el espectro, i.e. todos los niveles son distintos (momentum de espectro continuo),
Si el grupo es no conmutativo, entonces admite representaciones irreducibles de dimensión $d\geq 1$ y esperamos degeneraciones en el espectro de multiplicidad $d$. (i.e. distintos estados con misma energía).

\textbf{Nota:}
Esta propiedad resulta muy útil en física molecular: conociendo las representaciones irreducibles de distintos grupos (hay al menos 230 grupos finitos del espacio catalogados y observados en la naturaleza) y observando el espectro de una molécula, deducimos el grupo de invarianza y así podemos deducir la forma de la molécula (geométrica).

Espectro $\Rightarrow $ grupo de simetría $\Rightarrow$ forma de la molécula.

\textbf{Ejemplo: El espectro del átomo de hidrógeno}

\textit{Repaso:}

\begin{align*}
    \hat{H}=\frac{\hat{\vec{p}}^2}{2m}-\frac{e^2}{4\pi\epsilon_o}\frac{1}{r} 
\end{align*}

por el momento no tomamos en cuenta el espín. $\hat{H}$ es invariante ante rotaciones: $\left[\hat{H},\hat{T}\right]=0$. El espectro es tal que:

\begin{align*}
    \hat{H}\ket{\psi_{n,l,m}}=E_n\ket{\psi_{n,l,m}}   
\end{align*}


los autovectores son el producto de una función radial y de armónicos esféricos:

\begin{align*}
    \bra{x}\ket{\psi_{n,l,m}}=R_{n,l}(r)Y_{l,m}(\theta,\psi)
\end{align*}

Con $E_n=$

Cada uno de los niveles de energía constituye entonces un espacio propio 

\begin{align*}
    H_n=\bigoplus_{l=0}^{n-1}D_l
\end{align*}

Este espacio es de dimensión $n^2$

$$
\left(\sum_{l=0}^{n-1}(2l+1)=n^2\right)
$$

\textbf{Simetría adicional de Pauli, y degeneración a $l$:}

Esto pareciera contradecir el teorema de Wigner

De hecho no lo contradice, ya que el potencial tiene una forma muy particular, está en términos de $1/r$ y esto hace que haya una simetría más (descubierta por Pauli en $1925$), cuyo generador es:

\begin{align*}
    \hat{\vec{A}}=\frac{1}{2m}\left(\vec{p}\times\vec{L}-\vec{L}\times\vec{p}\right)-mK\frac{\vec{x}}{\abs{x}}
\end{align*}

con $K=\frac{e^2}{4\pi\epsilon}$ y $\vec{A}$ se conoce como el vector de Runge Lenz.

Es posible verificar que:

\begin{align*}
    \left[A,H\right]=0
\end{align*}

El grupo de simetrías es ahora $SO(4)$ de dimensión $6$ y cada espacio propio $H$ es irreducible para esta simetría.

Podemos romper esta simetría: correcciones relativistas. (rompen la simetría de $A$ pero no la invarianza rotacional)

\bigskip

\textbf{Átomo con varios electrones}

El espectro de átomos con un solo electrón ($z=2\leftarrow He^+$, $z=3\leftarrow Li^{2+}$) es lo mismo que $H$ pero sustituyendo la carga del núcleopor $Z_e$.

Ahora, si tenemos varios electrones, la situación es mucho más complicada: NO EXISTE SOLUCIÓN EXACTA. Pero se utilizan métodos de aproximación (método variacional).

Otro método: \textit{mean field theory} (método del campo medio). Considera que cada electrón es independiente pero está sometido a un potencial $V(r)$ promedio, resultado de todas las cargas: el núcleo $Z_e$ y los demás electrones.

Este potencial tiene simetría esférica, usualmente difiere de $1/r$. Esta diferencia rompe la simetría de Pauli. $\Rightarrow$ Se quita la degeneración de los estados con $l$ distinto.

\bigskip

\textbf{Efecto de un campo magnético externo:}
Tenemos un campo magnético externo $\vec{B}$: rompe la simetría esférica: le da prioridad a una dirección particular $\hat{B}$.

$\Rightarrow $ la degeneración de $m$ ya no va.

$\Rightarrow$ espectro no degenerado: efecto Zeeman.

Si el campo está en la dirección de $z$, hay una simetría de rotación en $z$.

Sin embargo, este subgrupo de simetrías forma un grupo conmutativo $\Rightarrow$ rompe la degeneración.

\bigskip

\textbf{Estructura fina e hiperfina del átomo de hidrógeno:}

El átomo de Bohr es una excelente aproximación, pero hay pequeñas correcciones por hacer.

Estos se detectaron gracias a la espectroscopía: se miden los niveles de energía con mucha precisión.

Estas correctiones se obtienen por métodos perturbativos y son mucho más pequeñas que $\varepsilon=13.6eV$.

\textit{Estructura fina:}
\begin{itemize}
    \item agregar una corrección relativista (viene de la ecuación de Dirac):
    $$
    \hat{H}_1'=-\frac{p^4}{8m^3c^2}+\frac{\pi\hbar^2}{2(mc)^2}\left(\frac{e^2}{4\pi\epsilon_o}\right)\delta(\vec{x})
    $$
    \item agregar la interacción del espín con la órbita:
    $$
    \hat{H}_2'=\frac{1}{2(mc)^2}\frac{1}{r}\dv*{v}{r}\vec{L}\cdot\vec{S}
    $$
\end{itemize}

Notemos que estas correcciones no rompen la simetría de rotación. 

$\hat{H}_2'$ enlaza a la posición con el espín. $\Rightarrow$ sólo el momentum angular total se conserva: $\vec{J}=\vec{L}+\vec{S}$.

Los resultados obtenidos dependen de $n,j$ y $l$ que corresponden a autovalores de $\vec{L}^2,\vec{J}^2$ que conmutan entre ellos y con $\hat{H}$.

\textit{Estructura hiperfina:}

Viene de enlazar los momentos magnéticos del protón y el electrón.



\section{Composición de momentum angulares}

\subsection{Partícula compuesta por dos partículas de espín $1/2$}

Por ejemplo, el núcleo del deuterio: protón y neutrón.
Un pión $\pi$ formado por $2$ quarks $\bar{\nu},\nu$ 
Un átomo de hidrógeno: protón $+$ electrón.

El espacio total de espín es:

\begin{align*}
    \mathbb{H}_{tot}=\mathbb{D}_{1/2}\otimes \mathbb{D}_{1/2}
\end{align*}

Con $\mathbb{D}_{1/2}:=\mathbb{H}_{spin}:=\mathbb{C}^2$. la bae del espacio es $\ket{+_z},\ket{-_z}$

En el espacio $\mathbb{H}_{tot }$ denotamos $\ket{++}=\ket{+_z}_1\otimes\ket{+_z}_2$.

Una base con las combinaciones de los elementos de las bases de cada partícula.

Si el sistema está aislado de su entorno (no interactúa con la órbita) $\Rightarrow$ es invariante ante rotaciones. El generador es el momentum angular total:

\begin{align*}
    \hat{\vec{S}}=\hat{\vec{S}}_1+\hat{\vec{S}}_2
\end{align*}


Y aquí, el hamiltoniano conmuta con $\hat{\vec{S}}$.

\begin{align*}
    \left[H,\vec{S}\right]=0
\end{align*}

Por ejemplo:

\begin{align*}
    \hat{H}=K\vec{S}_1\cdot \vec{S}_2
\end{align*}

De esta invarianza deducimos:

\begin{align*}
    \left[H,\vec{S}^2\right]&=0\\
    \left[H,S_z\right]&=0\\
    \left[\vec{S}^2,S_z\right]&=0\\
\end{align*}

Todos estos operadores tienen autovectores en común:

\begin{align*}
    \ket{J,M}
\end{align*}

Son tales que:

\begin{align*}
    \hat{S}_z\ket{J,M}&=M\hbar\ket{J,M}\\
    \hat{\vec{S}}^2\ket{J,M}&=\hbar^2 J\left(J+1\right)\ket{J,M}\\
\end{align*}

Objetivo: expresar la base $\ket{J,M}$ en función de $\ket{\pm,\pm}$.

\textbf{Propiedad:} Esta es la descomposición del espacio $\mathbb{H}_{tot}$ en vectores $\ket{J,M}$ ortonormales.

\begin{itemize}
    \item \textbf{Singlete}
    $$\ket{J=0,M=0}=\frac{1}{\sqrt{2}}\left(\ket{+-}-\ket{-+}\right)$$
    \item \textbf{Triplete}
    \begin{align*}
        \ket{J=1,M=1}&=\ket{++}\\
        \ket{J=1,M=0}&=\\
        \ket{J=1,M=-1}&=\ket{--}\\
    \end{align*}
\end{itemize}

Con esto, el espacio $\mathbb{H}_{tot}$ se descompone en la suma de 2 representaciones irreducibles del grupo de rotaciones:

$$\mathbb{D}_{1/2}\otimes\mathbb{D}_{1/2}=\mathbb{D}_{j=0}\oplus\mathbb{D}_{J=1}$$

Llamada descomposición de Clebsch - Gordon. La dimensión de los espacios es:

\begin{align*}
    2\times2=1+3
\end{align*}

\textbf{Nota}: Notemos que los valores posibles de $J$ de la partícula compuesta son la suma y la resta:

\begin{align*}
    J=\abs{\frac{1}{2}+\frac{1}{2}}=1
\end{align*}

Los coeficientes que acompañan a $\ket{\pm\pm}$ en la expresión del triplete y singlete se llaman coeficientes de Clebsch Gordon.


\includegraphics{Graficas/G1-Aug9.png}


