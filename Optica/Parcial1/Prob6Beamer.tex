\documentclass{beamer}
\usetheme{metropolis}           % Use metropolis theme
\title{Primera exposición}
\date{Viernes 30 de julio de 2021}
\author{Jorge Alejandro Rodríguez Aldana}
\institute{Escuela de Ciencias Físicas y Matemáticas}

\usepackage{physics}


\begin{document}
  \maketitle
  \section{Desplazamiento axial de un objeto proyectado por una lente}
  \begin{frame}{Problema 6:}
    % An object is located at a distance U to the left of a
    % thin lens and its image is formed a distance V to the
    % left of the same lens (see Figure II.15). If now the
    % object is displaced axially a small distance dU to the
    % left, find an expression for the corresponding dis-
    % placement dV of the image; dVldU is called the long-
    % itudinal magnification. Show that it is the square of
    % the lateral magnification.

    Un objeto es ubicado a una distancia $U$ a la izquierda de una
    lente delgada, y su imagen es formada a una distancia $V$ a la
    izquierda del mismo lente. Si ahora el 
    objeto se desplaza axialmente una pequeña distancia $\dd U$ a la
    izquierda, halle la expresión para el desplazamiento correspondiente
    $\dd V$ de la imagen. $\dv{V}{U}$ es llamada "magnificación longitudinal". 
    Muestre que esta es igual al cuadrado de la magnificación lateral.

    \begin{figure}
        \includegraphics[width=0.3\textwidth]{Figures/Book.png}
        \caption{Problema 6}
        \label{fig-1}
    \end{figure}
  \end{frame}
\end{document}