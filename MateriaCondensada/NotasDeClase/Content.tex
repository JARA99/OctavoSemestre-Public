\chapter{Introducción}

\section{¿Qué es la materia condensada?}

\begin{itemize}
    \item Fases condensadas: aparecen cuando los sistemas f\'isicos est\'an formados por un n\'umero grande de elementos que interact\'uan fuertemente.
    \item Fases condensadas bastante conocidas: s\'olidos, l\'iquidos.
    \item Otras fases condensadas: superconductores, superfluidos, ferromagnetos, antiferromagnetos, condensados Bose-Einstein.
    \item Otras mesofases: cristales líquidos, membranas autoensambladas, geles, coloides, cristales, vidrios, etc.
\end{itemize}

Categorías de la física:
\begin{itemize}
    \item Teórico
    \item Experimental
    \item Fenomenológica
    \item Computacional
\end{itemize}

\subsection{De la física del estado sólido a la física de la materia condensada}

\begin{itemize}
    \item La física del estado sólido en los 30 (siglo XX):
    \begin{itemize}
        \item Cristalografía por Rayos X
        \item Difracción de electrones
        \item M cuántica + M estadística
    \end{itemize}
\end{itemize}

