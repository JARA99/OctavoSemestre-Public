\documentclass[11pt]{article}
%%%%%%%%%%%%%%%%%%%%%%%%%%%%%%%% Algunos Paquetes Necesarios 
\usepackage{fancyhdr, graphicx, wrapfig,lipsum}
\usepackage[utf8]{inputenc} % Tildes
\usepackage[spanish]{babel} % Language
\usepackage{babelbib} % Bibliografia español
\usepackage[margin=1in]{geometry} % Margins														
\usepackage{amssymb}
\usepackage{amsmath, amsthm, amsfonts}
\usepackage[table]{xcolor} % Color table
\usepackage{longtable} % Table accross multiple pages
\usepackage{hyperref}  % Use Hyperlinks
\usepackage{enumerate} % Reduce space in enumerate
\usepackage{txfonts}
\setlength{\parindent}{0in}
\decimalpoint

\usepackage{cancel}

%%%%%%%%%%%%%%%%%%%%%%%%%%%%%%%%%%%%%%%%%%%%%%%%%%%%%%%%%%%%%%%%%%%%%%%%%%%%%%%%%%%%%%%%%%%%%%%%%%%%%%%%%%%%%%%%%%%%%%%%%%%%%%%%%%%%%%%%%%%%%%%%%%%%%%%%%%%%%%%%

%%%%%%%%%%%%%%%%%%%%%%%%%%%%%%%%%%%%%%%%%%%%%%%%%%%%%%%%%%%%%%%%%%%%%%%%%%%%%%%%%%%%%%%%%%%%%%%%%%%%%%%%%%%%%%%%%%%%%%%%%%%%%%%%%%%%%%%%%%%%%%%%%%%%%%%%%%%%%%%%
\newcommand{\myName}{Elementos del estilo \--- Resumen }
\newcommand{\myDate}{Guatemala, 16 de julio de 2021}
\newcommand{\myCourse}{Materia Condensada 1}


%\newcommand{\R}{\mathbb{R}}
%\newcommand{\F}{\mathbf{F}}
%\newcommand{\vi}{\mathbf{\hat{i}}}
%\newcommand{\vj}{\mathbf{\hat{j}}}
%\newcommand{\vk}{\mathbf{\hat{k}}}
%\newcommand{\op}{\sigma\sqrt{2\pi}}
%%%%%%%%%%%%%%%%%%%%%%%%%%%%%%%%%%%%%%%%%%%%%%%%%%%%%%%%%%%%%%%%%%%%%%%%%%%%%%%%%%%%%%%%%%%%%%%%%%%%%%%%%%%%%%%%%%%%%%%%%%%%%%%%%%%%%%%%%%%%%%%%%%%%%%%%%%%%%%%%

%%%%%%%%%%%%%%%%%%%%%%%%%%%%%%%%%%%%%%%%%%%%%%%%%%%%%%%%%%%%%%%%%%%%%%%%%%%%%%%%%%%%%%%%%%%%%%%%%%%%%%%%%%%%%%%%%%%%%%%%%%%%%%%%%%%%%%%%%%%%%%%%%%%%%%%%%%%%%%%%
%%%%%%%%%%%%%%%%%%%%%%%%%%%%%%%%%%% Tema - BEGIN
\newtheoremstyle{Tema}% name of the style to be used
  {5mm}% measure of space to leave above the theorem. E.g.: 3pt
  {10mm}% measure of space to leave below the theorem. E.g.: 3pt
  {}% name of font to use in the body of the theorem
  {}% measure of space to indent
  {\bfseries}% name of head font
  {\newline}% punctuation between head and body
  {30mm}% space after theorem head
  {}% Manually specify head

\theoremstyle{Tema} \newtheorem{Tema}{Tema} %%%%% Template para Temas
\theoremstyle{Tema} \newtheorem{Serie}{Serie}              %%%%%  Template para Series de ejercicios
\theoremstyle{Tema} \newtheorem{Ejercicio}{Ejercicio}    %%%%%  Template para Ejercicios
%%%%%%%%%%%%%%%%%%%%%%%%%%%%%%%%%%% Tema - END


%%%%%%%%%%%%%%%%%%%%%%%%%%%%%%%%%%% Encabezado - BEGIN %%%%%%%%%%
\fancypagestyle{firststyle}
{
\renewcommand{\headrulewidth}{1.5pt}
\fancyhead[R]{
			\textbf{Universidad de San Carlos de Guatemala} \\
			\textbf{Escuela de Ciencias Físicas y Matemáticas}\\
			\textbf{\myCourse }  \\  %%%%%%%%%% Agregar nombre del curso 
			\textbf{\myDate}   %%%%%%%%%%%%%%%%%%%%%% Agregar fecha en formato: Enero 15, 2015
			}
\fancyhead[L]{ 
	\includegraphics[height=1.6 cm]{/home/jorgealejandro/Templates/ECFM.png} \\
	\textbf{Jorge Alejandro Rodriguez Aldana}\\
	\textbf{201804766} 
	}
}
%%%%%%%%%%%%%%%%%%%%%%%%%%%%%%%%%%% Encabezado - END %%%%%%%%%%

%%%%%%%%%%%%%%%%%%%%%%%%%%%%%%%%%%% Encabezado (pagina 2 en adelante) - BEGIN %%%
\fancypagestyle{allStyle}
{
\renewcommand{\headrulewidth}{1pt}
\fancyhead[R]{
			\emph{\myName $-$ \myCourse} %%%% Modificar número de examen parcial y nombre del curso
			}
\fancyhead[L]{}  
\fancyfoot[C]{}
\fancyfoot[R]{\thepage}
}
%%%%%%%%%%%%%%%%%%%%%%%%%%%%%%%%%%% Encabezado (pagina 2 en adelante) - END %%%

\date{}
\setlength{\headheight}{0.8in} % fixes \headheight warning

%%%%%%%%%%%%%%%%%%%%%%%% BEGIN %%%%%%%%%%%%%%
\begin{document}

\pagestyle{allStyle}

\thispagestyle{firststyle}
%%%%%%%%%%%%%%%%%%%%%%%%%%%%%%%%% Titulo - BEGIN
\begin{center}
\LARGE
\textsc{\myName} % Modificar el Número del examen parcial
\medskip
\hrule height 1.5pt
\end{center}
%%%%%%%%%%%%%%%%%%%%%%%%%%%%%%%%%%%%%%%%%%%%%%%%%%%%%%%%%%%%%%%%%%%%%%%%%%%%%%%%%%%%%%%%%%%%%%%%%%%%%%%%%%%%%%%%%%%%%%%%%%%%%%%%%%%%%%%%%%%%%%%%%%%%%%%%%%%%%%%%


%%%%%%%%%%%%%%%%%%%%%%%%%%%%%%%%% Instrucciones - BEGIN
\vspace{0.1 in}

Este artículo da una idea sobre el tipo de redacción que se busca cuando se está escribiendo para la revista \textit{Nature Physics. ``We regularly get queries about the minutiae of Nature Physics format, but what we really care about is that the papers are clear and accessibly written.''}\cite{NatElementsOfStyle}. Como bien lo dice la frase, lo que ellos buscan es una redacción clara y accesible para los lectores. Cada vez que se acepta un \textit{paper} para publicar, este es revisado, y la mayoría de las veces se hacen cambios en el estilo del texto.

\medskip

Comienza describiendo la importancia de un buen título, y haciendo una pregunta que me parece bastante buena  ``¿Por que a alguien le importaría leer mi \textit{papaer} pasado el título?''\cite{NatElementsOfStyle}. Aquí es donde vemos la importancia de un título llamativo e informativo (aunque más adelante habla sobre no exagerar con el fin de retener lectores).
Después, le da un papel importante al primer párrafo, en el que se debe introducir y poner en contexto al lector. No es buena idea comenzar con la teoría y los métodos utilizados desde el principio, cualquier lector agradecerá que se le ponga en contexto del tema antes de comenzar a hablar de este.

\medskip

Debemos contar una historia con nuestro artículo, y aunque los detalles técnicos son importantes para la ciencia, no lo son para la narrativa. También se debe evitar el uso de prefijos innecesarios.
Es importante no exagerar con el fin de emocionar al lector.

\medskip

Por último, habla sobre concluir las ideas, el último párrafo debe asentar lo aprendido por el lector, pero no debe ser un resumen de todo el párrafo.

\bibliographystyle{ieeetr}% otras opciones {plain}
\bibliography{Ref1}

\end{document} %%%%%%%%%%%%%%%%%%%%%%%% BEGIN%%%%%%%%%%%%%%