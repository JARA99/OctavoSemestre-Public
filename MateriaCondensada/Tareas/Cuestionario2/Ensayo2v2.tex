% 20200720
% Giovanni Ramirez
% ramirez@ecfm.usac.edu.gt

%
% Esta es una plantilla y contiene lo mínimo que debe llevar un artículo,
% ensayo o tarea corta para el curso de Física de la Materia Condensada 1.
% Esta plantilla se puede usar tanto en un compilador de LaTeX en su
% computadora o en algún compilador en línea como overleaf.com
%

\documentclass[11pt]{article}

%%% PAQUETES
% esto es para hacer las cosas en españo y para usar el punto como separador
% decimal
\usepackage[spanish,es-nodecimaldot]{babel}%
% esto es por la codificación con la que se escribe, revisar la codificación
% de su sistema pues podría estar en utf16 o en iso-latin-1 o iso 8859-1
\usepackage[utf8]{inputenc}%
% esto es para agregar gráficas y figuras
\usepackage{graphicx}%
% esto es por algunos símbolos como la ñ o las vocales tildadas
\usepackage{latexsym}%
% esto es para tener acceso a varios símbolos matemáticos
\usepackage{amsfonts, amsmath}%
% esto es para poder usar el sistema internacional de unidades
\usepackage[amssymb]{SIunits}%
% esto es para que la página sea tamaño carta
\usepackage[letterpaper]{geometry}
% esto es útil para los hipervínculos
\usepackage{hyperref}

% aquí va el título
\title{Guía para escribir un ensayo de física \--- Resumen}
% aquí el autor
\author{Jorge Alejandro Rodriguez Aldana}
% aquí la fecha
\date{Guatemala, 16 de julio de 2021}

\begin{document}
\maketitle

% \begin{abstract}
%   Aquí hay que poner el resumen: la idea es que una persona lea el resumen y
%   sepa de qué se trata el ensayo o artículo, también que sepa qué método se
%   usa para sacar los resultados y cuáles son los resultados más
%   importantes. Para que las referencias salgan hay que citarlas, por ejemplo
%   para incluir el libro de Ashcroft y Mermin \cite{ashcroft}.
% \end{abstract}


Muchos ven el proceso para escribir un ensayo como algo difícil, pero no tiene por que ser así necesariamente. 
Antes de escribir un ensayo, debemos definir qué es un ensayo: ``un ensayo es un tipo de escritura que es usualmente más corta que un artículo y debe ser más fácil de leer para un nivel menos profesional.'' \cite{Guide}
De cualquier modo, debe cumplir algunos requisitos.

\medskip

Para escoger un tema, se necesita un tiempo de investigación, y la elección de un tema interesante. Plantear el tema como una pregunta puede ser una buena idea.

\medskip

Existen distintos tipos de ensayos. Por ejemplo el descriptivo, que nos da una explicación detallada de un tema en específico. El de comparación y contraste, el cual nos enseña las similitudes y diferencias de los conceptos. El de proceso, usualmente es un experimento y el análisis de sus resultados. El narrativo, que es una descripción histórica. El de causa y efecto, que es una secuencia de eventos. Y el argumentativo, que trata de persuadir al lector de aceptar la posición del escritor.

\medskip

Podemos acomodar estos ensayos en tres grupos: Histórico, Física Fundamental e Investigación Moderna. 

Cada uno de estos ensayos dicta un conocimiento básico del tema.

Se debe ser preciso con los conceptos, si se necesita apoyo de gráficas o ecuaciones, deben ser escritas en un formato adecuado. Se debe citar los trabajos que sean copiados o parafraseados.

\medskip

Una estructura adecuada para un ensayo es la siguiente:

\begin{enumerate}
	\item Título 
	\item Tesis
	\item Introducción
	\item Cuerpo
	\item Conclusión
\end{enumerate}

\medskip

Por último, el escritor debe revisar su trabajo terminado, y corregir errores de lógica y continuidad. Este proceso puede ser desafiante, pero una estrategia efectiva es usar una \textit{``checklist''} para la revisión.


\bibliographystyle{ieeetr}% otras opciones {plain}
\bibliography{Ref2}


\end{document}

