\documentclass[11pt]{article}
%%%%%%%%%%%%%%%%%%%%%%%%%%%%%%%% Algunos Paquetes Necesarios 
\usepackage{fancyhdr, graphicx, wrapfig,lipsum}
\usepackage[utf8]{inputenc} % Tildes
\usepackage[spanish]{babel} % Language
\usepackage{babelbib} % Bibliografia español
\usepackage[margin=1in]{geometry} % Margins														
\usepackage{amssymb}
\usepackage{amsmath, amsthm, amsfonts}
\usepackage[table]{xcolor} % Color table
\usepackage{longtable} % Table accross multiple pages
\usepackage{hyperref}  % Use Hyperlinks
\usepackage{enumerate} % Reduce space in enumerate
\usepackage{txfonts}
\setlength{\parindent}{0in}
\decimalpoint

\usepackage{cancel}

%%%%%%%%%%%%%%%%%%%%%%%%%%%%%%%%%%%%%%%%%%%%%%%%%%%%%%%%%%%%%%%%%%%%%%%%%%%%%%%%%%%%%%%%%%%%%%%%%%%%%%%%%%%%%%%%%%%%%%%%%%%%%%%%%%%%%%%%%%%%%%%%%%%%%%%%%%%%%%%%

%%%%%%%%%%%%%%%%%%%%%%%%%%%%%%%%%%%%%%%%%%%%%%%%%%%%%%%%%%%%%%%%%%%%%%%%%%%%%%%%%%%%%%%%%%%%%%%%%%%%%%%%%%%%%%%%%%%%%%%%%%%%%%%%%%%%%%%%%%%%%%%%%%%%%%%%%%%%%%%%
\newcommand{\myName}{Guía para escribir un ensayo de física \--- Resumen }
\newcommand{\myDate}{Guatemala, 16 de julio de 2021}
\newcommand{\myCourse}{Materia Condensada 1}


%\newcommand{\R}{\mathbb{R}}
%\newcommand{\F}{\mathbf{F}}
%\newcommand{\vi}{\mathbf{\hat{i}}}
%\newcommand{\vj}{\mathbf{\hat{j}}}
%\newcommand{\vk}{\mathbf{\hat{k}}}
%\newcommand{\op}{\sigma\sqrt{2\pi}}
%%%%%%%%%%%%%%%%%%%%%%%%%%%%%%%%%%%%%%%%%%%%%%%%%%%%%%%%%%%%%%%%%%%%%%%%%%%%%%%%%%%%%%%%%%%%%%%%%%%%%%%%%%%%%%%%%%%%%%%%%%%%%%%%%%%%%%%%%%%%%%%%%%%%%%%%%%%%%%%%

%%%%%%%%%%%%%%%%%%%%%%%%%%%%%%%%%%%%%%%%%%%%%%%%%%%%%%%%%%%%%%%%%%%%%%%%%%%%%%%%%%%%%%%%%%%%%%%%%%%%%%%%%%%%%%%%%%%%%%%%%%%%%%%%%%%%%%%%%%%%%%%%%%%%%%%%%%%%%%%%
%%%%%%%%%%%%%%%%%%%%%%%%%%%%%%%%%%% Tema - BEGIN
\newtheoremstyle{Tema}% name of the style to be used
  {5mm}% measure of space to leave above the theorem. E.g.: 3pt
  {10mm}% measure of space to leave below the theorem. E.g.: 3pt
  {}% name of font to use in the body of the theorem
  {}% measure of space to indent
  {\bfseries}% name of head font
  {\newline}% punctuation between head and body
  {30mm}% space after theorem head
  {}% Manually specify head

\theoremstyle{Tema} \newtheorem{Tema}{Tema} %%%%% Template para Temas
\theoremstyle{Tema} \newtheorem{Serie}{Serie}              %%%%%  Template para Series de ejercicios
\theoremstyle{Tema} \newtheorem{Ejercicio}{Ejercicio}    %%%%%  Template para Ejercicios
%%%%%%%%%%%%%%%%%%%%%%%%%%%%%%%%%%% Tema - END


%%%%%%%%%%%%%%%%%%%%%%%%%%%%%%%%%%% Encabezado - BEGIN %%%%%%%%%%
\fancypagestyle{firststyle}
{
\renewcommand{\headrulewidth}{1.5pt}
\fancyhead[R]{
			\textbf{Universidad de San Carlos de Guatemala} \\
			\textbf{Escuela de Ciencias Físicas y Matemáticas}\\
			\textbf{\myCourse }  \\  %%%%%%%%%% Agregar nombre del curso 
			\textbf{\myDate}   %%%%%%%%%%%%%%%%%%%%%% Agregar fecha en formato: Enero 15, 2015
			}
\fancyhead[L]{ 
	\includegraphics[height=1.6 cm]{/home/jorgealejandro/Templates/ECFM.png} \\
	\textbf{Jorge Alejandro Rodriguez Aldana}\\
	\textbf{201804766} 
	}
}
%%%%%%%%%%%%%%%%%%%%%%%%%%%%%%%%%%% Encabezado - END %%%%%%%%%%

%%%%%%%%%%%%%%%%%%%%%%%%%%%%%%%%%%% Encabezado (pagina 2 en adelante) - BEGIN %%%
\fancypagestyle{allStyle}
{
\renewcommand{\headrulewidth}{1pt}
\fancyhead[R]{
			\emph{\myName $-$ \myCourse} %%%% Modificar número de examen parcial y nombre del curso
			}
\fancyhead[L]{}  
\fancyfoot[C]{}
\fancyfoot[R]{\thepage}
}
%%%%%%%%%%%%%%%%%%%%%%%%%%%%%%%%%%% Encabezado (pagina 2 en adelante) - END %%%

\date{}
\setlength{\headheight}{0.8in} % fixes \headheight warning

%%%%%%%%%%%%%%%%%%%%%%%% BEGIN %%%%%%%%%%%%%%
\begin{document}

\pagestyle{allStyle}

\thispagestyle{firststyle}
%%%%%%%%%%%%%%%%%%%%%%%%%%%%%%%%% Titulo - BEGIN
\begin{center}
\LARGE
\textsc{\myName} % Modificar el Número del examen parcial
\medskip
\hrule height 1.5pt
\end{center}
%%%%%%%%%%%%%%%%%%%%%%%%%%%%%%%%%%%%%%%%%%%%%%%%%%%%%%%%%%%%%%%%%%%%%%%%%%%%%%%%%%%%%%%%%%%%%%%%%%%%%%%%%%%%%%%%%%%%%%%%%%%%%%%%%%%%%%%%%%%%%%%%%%%%%%%%%%%%%%%%


%%%%%%%%%%%%%%%%%%%%%%%%%%%%%%%%% Instrucciones - BEGIN
\vspace{0.1 in}

Muchos ven el proceso para escribir un ensayo como algo difícil, pero no tiene por que ser así necesariamente. 
Antes de escribir un ensayo, debemos definir qué es un ensayo: ``un ensayo es un tipo de escritura que es usualmente más corta que un artículo y debe ser más fácil de leer para un nivel menos profesional.'' \cite{Guide}
De cualquier modo, debe cumplir algunos requisitos.

\medskip

Para escoger un tema, se necesita un tiempo de investigación, y la elección de un tema interesante. Plantear el tema como una pregunta puede ser una buena idea.

\medskip

Existen distintos tipos de ensayos. Por ejemplo el descriptivo, que nos da una explicación detallada de un tema en específico. El de comparación y contraste, el cual nos enseña las similitudes y diferencias de los conceptos. El de proceso, usualmente es un experimento y el análisis de sus resultados. El narrativo, que es una descripción histórica. El de causa y efecto, que es una secuencia de eventos. Y el argumentativo, que trata de persuadir al lector de aceptar la posición del escritor.

\medskip

Podemos acomodar estos ensayos en tres grupos: Histórico, Física Fundamental e Investigación Moderna. 

Cada uno de estos ensayos dicta un conocimiento básico del tema.

Se debe ser preciso con los conceptos, si se necesita apoyo de gráficas o ecuaciones, deben ser escritas en un formato adecuado. Se debe citar los trabajos que sean copiados o parafraseados.

\medskip

Una estructura adecuada para un ensayo es la siguiente:

\begin{enumerate}
	\item Título 
	\item Tesis
	\item Introducción
	\item Cuerpo
	\item Conclusión
\end{enumerate}

\medskip

Por último, el escritor debe revisar su trabajo terminado, y corregir errores de lógica y continuidad. Este proceso puede ser desafiante, pero una estrategia efectiva es usar una \textit{``checklist''} para la revisión.


\bibliographystyle{ieeetr}% otras opciones {plain}
\bibliography{Ref2}

\end{document} %%%%%%%%%%%%%%%%%%%%%%%% BEGIN%%%%%%%%%%%%%%