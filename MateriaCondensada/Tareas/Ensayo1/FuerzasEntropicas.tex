\documentclass[11pt]{article}
%%%%%%%%%%%%%%%%%%%%%%%%%%%%%%%% Algunos Paquetes Necesarios 
\usepackage{fancyhdr, graphicx, wrapfig,lipsum}
\usepackage[utf8]{inputenc} % Tildes
\usepackage[spanish]{babel} % Language
\usepackage{babelbib} % Bibliografia español
\usepackage[margin=1in]{geometry} % Margins														
\usepackage{amssymb}
\usepackage{amsmath, amsthm, amsfonts}
\usepackage[table]{xcolor} % Color table
\usepackage{longtable} % Table accross multiple pages
\usepackage{hyperref}  % Use Hyperlinks
\usepackage{enumerate} % Reduce space in enumerate
\usepackage{txfonts}
\setlength{\parindent}{0in}
\decimalpoint

\usepackage{cancel}

%%%%%%%%%%%%%%%%%%%%%%%%%%%%%%%%%%%%%%%%%%%%%%%%%%%%%%%%%%%%%%%%%%%%%%%%%%%%%%%%%%%%%%%%%%%%%%%%%%%%%%%%%%%%%%%%%%%%%%%%%%%%%%%%%%%%%%%%%%%%%%%%%%%%%%%%%%%%%%%%

%%%%%%%%%%%%%%%%%%%%%%%%%%%%%%%%%%%%%%%%%%%%%%%%%%%%%%%%%%%%%%%%%%%%%%%%%%%%%%%%%%%%%%%%%%%%%%%%%%%%%%%%%%%%%%%%%%%%%%%%%%%%%%%%%%%%%%%%%%%%%%%%%%%%%%%%%%%%%%%%
\newcommand{\myName}{Fuerzas entrópicas}
\newcommand{\myDate}{Guatemala, 16 de julio de 2021}
\newcommand{\myCourse}{Materia Condensada 1}


%\newcommand{\R}{\mathbb{R}}
%\newcommand{\F}{\mathbf{F}}
%\newcommand{\vi}{\mathbf{\hat{i}}}
%\newcommand{\vj}{\mathbf{\hat{j}}}
%\newcommand{\vk}{\mathbf{\hat{k}}}
%\newcommand{\op}{\sigma\sqrt{2\pi}}
%%%%%%%%%%%%%%%%%%%%%%%%%%%%%%%%%%%%%%%%%%%%%%%%%%%%%%%%%%%%%%%%%%%%%%%%%%%%%%%%%%%%%%%%%%%%%%%%%%%%%%%%%%%%%%%%%%%%%%%%%%%%%%%%%%%%%%%%%%%%%%%%%%%%%%%%%%%%%%%%

%%%%%%%%%%%%%%%%%%%%%%%%%%%%%%%%%%%%%%%%%%%%%%%%%%%%%%%%%%%%%%%%%%%%%%%%%%%%%%%%%%%%%%%%%%%%%%%%%%%%%%%%%%%%%%%%%%%%%%%%%%%%%%%%%%%%%%%%%%%%%%%%%%%%%%%%%%%%%%%%
%%%%%%%%%%%%%%%%%%%%%%%%%%%%%%%%%%% Tema - BEGIN
\newtheoremstyle{Tema}% name of the style to be used
  {5mm}% measure of space to leave above the theorem. E.g.: 3pt
  {10mm}% measure of space to leave below the theorem. E.g.: 3pt
  {}% name of font to use in the body of the theorem
  {}% measure of space to indent
  {\bfseries}% name of head font
  {\newline}% punctuation between head and body
  {30mm}% space after theorem head
  {}% Manually specify head

\theoremstyle{Tema} \newtheorem{Tema}{Tema} %%%%% Template para Temas
\theoremstyle{Tema} \newtheorem{Serie}{Serie}              %%%%%  Template para Series de ejercicios
\theoremstyle{Tema} \newtheorem{Ejercicio}{Ejercicio}    %%%%%  Template para Ejercicios
%%%%%%%%%%%%%%%%%%%%%%%%%%%%%%%%%%% Tema - END


%%%%%%%%%%%%%%%%%%%%%%%%%%%%%%%%%%% Encabezado - BEGIN %%%%%%%%%%
\fancypagestyle{firststyle}
{
\renewcommand{\headrulewidth}{1.5pt}
\fancyhead[R]{
			\textbf{Universidad de San Carlos de Guatemala} \\
			\textbf{Escuela de Ciencias Físicas y Matemáticas}\\
			\textbf{\myCourse }  \\  %%%%%%%%%% Agregar nombre del curso 
			\textbf{\myDate}   %%%%%%%%%%%%%%%%%%%%%% Agregar fecha en formato: Enero 15, 2015
			}
\fancyhead[L]{ 
	\includegraphics[height=1.6 cm]{/home/jorgealejandro/Templates/ECFM.png} \\
	\textbf{Jorge Alejandro Rodriguez Aldana}\\
	\textbf{201804766} 
	}
}
%%%%%%%%%%%%%%%%%%%%%%%%%%%%%%%%%%% Encabezado - END %%%%%%%%%%

%%%%%%%%%%%%%%%%%%%%%%%%%%%%%%%%%%% Encabezado (pagina 2 en adelante) - BEGIN %%%
\fancypagestyle{allStyle}
{
\renewcommand{\headrulewidth}{1pt}
\fancyhead[R]{
			\emph{\myName $-$ \myCourse} %%%% Modificar número de examen parcial y nombre del curso
			}
\fancyhead[L]{}  
\fancyfoot[C]{}
\fancyfoot[R]{\thepage}
}
%%%%%%%%%%%%%%%%%%%%%%%%%%%%%%%%%%% Encabezado (pagina 2 en adelante) - END %%%

\date{}
\setlength{\headheight}{0.8in} % fixes \headheight warning

%%%%%%%%%%%%%%%%%%%%%%%% BEGIN %%%%%%%%%%%%%%
\begin{document}

\pagestyle{allStyle}

\thispagestyle{firststyle}
%%%%%%%%%%%%%%%%%%%%%%%%%%%%%%%%% Titulo - BEGIN
\begin{center}
\LARGE
\textsc{\myName} % Modificar el Número del examen parcial
\medskip
\hrule height 1.5pt
\end{center}
%%%%%%%%%%%%%%%%%%%%%%%%%%%%%%%%%%%%%%%%%%%%%%%%%%%%%%%%%%%%%%%%%%%%%%%%%%%%%%%%%%%%%%%%%%%%%%%%%%%%%%%%%%%%%%%%%%%%%%%%%%%%%%%%%%%%%%%%%%%%%%%%%%%%%%%%%%%%%%%%


%%%%%%%%%%%%%%%%%%%%%%%%%%%%%%%%% Instrucciones - BEGIN
\vspace{0.1 in}

Para hablar sobre fuerzas entrópicas, primero debemos hablar de entropía.
A la entropía comúnmente se le asocia con desorden, sin embargo, la definición real parece partir de todo lo contrario, el orden.

En termodinámica la entropía se define como la magnitud física para un sistema termodinámico en equilibrio, que mide el número de microestados compatibles con el macroestado de equilibrio, o en otras palabras, es la magnitud que mide el grado de organización del sistema. La palabra entropía viene del griego $\varepsilon\nu\tau\rho o \pi i\alpha$ que significa evolución o transformación, y su valor crece en un sistema aislado, en un proceso que se da de forma natural. \cite{WikiEntropy}

\bigskip

La entropía está presente en cualquier sistema termodinámico (y por tanto, está presente en todos los sistemas de la vida real). Debido a que un sistema tenderá a aumentar el valor de su entropía, pueden aparecer fuerzas como fenómenos emergentes que resultan de esta tendencia; es a estas fuerzas a las que llamamos fuerzas entrópicas. \cite{HandWiki-EntropicForce}

Algunos ejemplos \cite{Wiki-EntropicForce} de fuerzas entrópicas son el movimiento Browniano, que es el movimiento aleatorio de particulars suspendidas en un medio \cite{Wiki-Brownian}. También la presión de un gas ideal, la cual depende únicamente de la temperatura del gas. La elasticidad de polímeros, la fuerza hidrofóbica, la fuerza de agotamiento en los coloides y las fuerzas contractivas del citoesqueleto también son buenos ejemplos de fuerzas entrópicas.

La forma matemática que se les da a estas fuerzas, está asociada a una partición $\{X\}$ del macroestado. De modo que la fuerza entrópica $F$ está dada por:

\begin{align}
	F(X_o)=T\Lambda_xS(X)\lvert_{X_o}
\end{align}

donde:

\begin{itemize}
	\item $T$ es la temperatura
	\item $S(X)$ es la entropía asociada al macroestado $X$
	\item $X_o$ es el presente macroestado
\end{itemize}

\hfill\cite{WikiEntropy}

\bigskip

Una aplicación interesante, aunque de momento, con pocas pruebas y en desarrollo, es la aplicación de fuerzas entrópicas a algo que en mi opinion, es un poco más subjetivo, la inteligencia. Estudios recientes sugieren una posible conexión entre la inteligencia y la maximización de la entropía \cite{PhysRevLett}, y como hemos visto ya, cuando la entropía se maximiza, pueden aparecer fuerzas espontáneas, así que de esto ser así, se podría estudiar a la inteligencia como una fuerza entrópica.

\bigskip

Es indiscutible que las fuerzas entrópicas están presentes en la física del mundo que conocemos. Pero, hasta donde podemos realmente modelar lo que percibimos como un sistema termodinámico, como delimitar que análisis es un fenómeno físico, y que es una mera interpretación. Será posible abstraer cualquier problema a uno de fuerzas entrópicas. Esta discusión puede ser un tanto más filosófica, pero es una discusión válida, de lo contrario, no existirían ejemplos controversiales de fuerzas entrópicas como el de la inteligencia o explicar la gravedad. 
Lo que es seguro, es que entender de forma general como se comportan las fuerzas entrópicas simplifica el entendimiento de cada una, aunque estén asociadas a sistemas completamente distintos.

\bibliographystyle{ieeetr}% otras opciones {plain}
\bibliography{Ref1}

\end{document} %%%%%%%%%%%%%%%%%%%%%%%% BEGIN%%%%%%%%%%%%%%